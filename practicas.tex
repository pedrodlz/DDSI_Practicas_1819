\documentclass[12pt,a4paper]{article}
\renewcommand{\baselinestretch}{1.05}
\usepackage[spanish]{babel}
%\usepackage[utf8]{inputenc}

\usepackage{amsmath,amsthm,verbatim,amssymb,amsfonts,amscd, graphicx}
\usepackage{graphicx}
\usepackage{caption}
\usepackage{subcaption}
\usepackage{tkz-fct}
\usetikzlibrary{babel}
\usepackage{pgfplots}
\usepackage{booktabs}
\usepackage{float}
\usepackage{enumitem}
\usepackage{forest}
\usepackage{hyperref}
\usepackage{fancybox}
\usepackage{listings}



%Uso de la fuente Source Sans Pro

\usepackage[default]{sourcesanspro}
%\usepackage[T1]{fontenc}

%Controlar la partición de palabras.
\pretolerance=5000
\tolerance=6000

%Simbolo de euro
\usepackage{eurosym} % para el euro


%Definición de monospace para codigo inline y paquete listings para código fuente.
\def\code#1{\texttt{#1}}
\usepackage{listingsutf8}
\lstset{
    %extendedchars=false,
    %inputencoding=utf8
}
\usepackage{color}
\definecolor{grisclarito}{gray}{0.95}

\lstdefinestyle{customc}{
  %belowcaptionskip=1\baselineskip,
  breaklines=true,
  frame=single,
  %xleftmargin=\parindent,
  language=C,
  showstringspaces=false,
  %basicstyle=\ttfamily,
  keywordstyle=\bfseries\color{green!40!black},
  commentstyle=\itshape\color{purple!40!black},
  identifierstyle=\color{blue},
  stringstyle=\color{orange},
  backgroundcolor=\color{grisclarito}
}

\lstdefinelanguage{JavaScript}{
keywords={typeof, new, true, false, catch, function, return, null, catch, switch, var, if, in, while, do, else, case, break},
keywordstyle=\color{blue}\bfseries,
ndkeywords={class, export, boolean, throw, implements, import, this},
ndkeywordstyle=\color{darkgray}\bfseries,
identifierstyle=\color{black},
sensitive=false,
comment=[l]{//},
morecomment=[s]{/*}{*/},
commentstyle=\color{purple}\ttfamily,
stringstyle=\color{red}\ttfamily,
morestring=[b]',
morestring=[b]"
}

\hypersetup{
  colorlinks=true,
  linkcolor=black,
  urlcolor=blue
}

\setcounter{secnumdepth}{5}

\setlength{\parindent}{0pt}
\topmargin0.0cm
\headheight0.0cm
\headsep0.0cm
\oddsidemargin0.0cm
\textheight23.0cm
\textwidth16.5cm
\footskip1.0cm

\renewcommand*\contentsname{Índice}
\newcommand{\mylabel}[2]{#2\def\@currentlabel{#2}\label{#1}}

\begin{document}
\begin{titlepage}
  \centering
  \includegraphics[width=0.6\textwidth]{imagenes/ugr.png}\par\vspace{1cm}
  {\scshape\large Diseño y Desarrollo de Sistemas de Información \par} \vspace{1cm}
  {\huge\bfseries Diseño de un sistema de información para la gestión y reproducción de música. \par}
  \vspace{0.4cm}
  {\large\itshape Entrega final.}
  \vspace{0.6cm}
  {\large\itshape \\ Darío Abad Tarifa \\ Juan Francisco Díaz Moreno \\ Pedro Domínguez López \\ Javier Sáez de la Coba \par} \vspace{1.00cm}
  Curso 2018-2019 \\
  \vfill

  % Bottom of the page
  {\large \today\par}
\end{titlepage}

\tableofcontents
\newpage

\setlength{\parskip}{10pt}

\section{Descripción del problema a resolver.}

Una empresa de streaming de audio quiere rehacer su plataforma de gestión de música (todo el servicio excepto el propio streaming de música). Para ello requiere que puedan haber cuentas de oyentes y de artistas.

El sistema ofrece un servicio de música en streaming, es decir permite que un usuario pueda buscar una canción, un álbum o una lista de música en la base 
de datos para posteriormente escucharla. 

Las listas funcionan como si de un directorio se tratase, el usuario las gestiona como quiere y en ellas guarda sus canciones favoritas o canciones según un tema especifico.  

El sistema ofrece ademas diferentes opciones de carácter social para facilitar la relación entre amigos y los artistas con sus oyentes. Un usuario puede seguir a otro usuario (sea artista o no) para estar al corriente de lo que escucha proporcionando el nombre del usuario al que quiere seguir. Un usuario podrá recomendar una canción a otro usuario proporcionando el nombre de este y el identificador de la canción que quiere recomendar. Un usuario podrá solicitar un resumen sobre las canciones que escucha su lista de amigos. 
Un usuario podrá ver las recomendaciones que le llegan de sus amigos. Además, los usuarios oyentes pueden valorar las canciones que escuchan. Por último podrán visualizar las canciones mejor valoradas por un usuario proporcionando su nombre de usuario.

\section{Análisis de requisitos.}

\subsection{Requisitos de datos}

\begin{enumerate}[label=\textnormal{RD\arabic*}]

	\item Identificador de la canción, álbum o lista para escuchar: \label{rd1}
		\begin{itemize}
			\item Elección (canción, álbum o lista)
			\item Identificador
		\end{itemize}
		
	\item Datos de álbum almacenado: \label{rd2}
		\begin{itemize}
			\item Nombre del álbum
			\item Nombre del artista
			\item Fecha de introducción
			\item Identificador del álbum
        	\item Nombre del álbum
		    \item Nombre del artista
		    \item Fecha de introducción
			\item Número de canciones
			\item Duración
		\end{itemize}
		
		
	\item Datos de canción almacenada: \label{rd3}
		\begin{itemize}
			\item Identificador de la canción
		    \item Nombre de la canción
		    \item Nombre del artista
		    \item Nombre del álbum
		    \item Estilo de la canción
		    \item Duración de la canción
		    \item Fecha de introducción
		    \item Archivo de audio
		    \item Número de reproducciones
		    \item Valoración media
		\end{itemize}
		
	\item Datos de lista almacenada: \label{rd4}
		\begin{itemize}
		    \item Identificador de la lista
		    \item Nombre de la lista
		    \item Canciones que contiene
		    \item Duración de la lista
		    \item Fecha de creación
		    \item Usuario al que pertenece
		    \item Seguidores
		\end{itemize}
		
	\item Identificador de la nueva lista creada: \label{rd5}
		\begin{itemize}
			\item Identificador de la lista
		\end{itemize}
		
	\item Nombre para la búsqueda de una canción, álbum o lista: \label{rd6}
		\begin{itemize}
			\item Nombre
		\end{itemize}
		
	\item Lista con los posibles identificadores según el resultado de la búsqueda: \label{rd7}
		\begin{itemize}
			\item Identificador
			\item Tipo: canción, álbum, lista
			\item Nombre
			\item Usuario o artista
			\item Fecha de publicación o de creación
			\item Duración
		\end{itemize}
		
	\item Datos para crear una lista nueva: \label{rd8}
		\begin{itemize}
			\item Usuario
			\item Fecha 
		\end{itemize}
		
	\item Valoración de una canción: \label{rd9}
		\begin{itemize}
			\item Identificador de la canción
			\item Identificador
		\end{itemize}
		
	\item Datos nuevo álbum: \label{rd10}
		\begin{itemize}
		    \item Nombre del álbum
		    \item Nombre del artista
		    \item Fecha de introducción
		\end{itemize}
		
	\item Datos nueva canción: \label{rd11}
		\begin{itemize}
		    \item Nombre de la canción
		    \item Nombre del artista
		    \item Nombre del álbum
		    \item Estilo de la canción
		    \item Duración de la canción
		    \item Fecha de introducción
		    \item Archivo de audio
		\end{itemize}
		
	\item  Identificador de la canción introducida: \label{rd12}
		\begin{itemize}
			\item Identificador de la canción
		\end{itemize}
		
	\item Identificador de la canción para ver sus estadísticas: \label{rd13}
		\begin{itemize}
			\item Identificador de la canción
		\end{itemize}
		
	\item Reproduccion de una lista: \label{rd14}
		\begin{itemize}
		   	\item Audio de las canciones de la lista
		\end{itemize}
		
	\item Estadísticas de la canción: \label{rd15}
		\begin{itemize}
		    \item Nombre de la canción
		    \item Número de reproducciones
		    \item Valoración media
		\end{itemize}
		
	\item Identificador de la canción para añadirla a canciones destacadas: \label{rd16}
		\begin{itemize}
			\item Identificador de la canción
		\end{itemize}
		
	\item Lista de las canciones destacadas actualizada: \label{rd17}
		\begin{itemize}
			\item Canciones destacadas
		\end{itemize}
		
	\item Identificador de la canción o álbum para eliminarlo: \label{rd18}
		\begin{itemize}
			\item Elección (canción o álbum)
			\item Identificador
		\end{itemize}
		
	\item Datos de registro del usuario: \label{rd19}
		\begin{itemize}
			\item Nombre de usuario
			\item Correo electrónico
			\item Contraseña
			\item Nombre
			\item Apellidos
			\item Tipo de usuario (oyente, artista)
			\item Dirección
		\end{itemize}
		
	\item Datos usuario almacenado: \label{rd20}
		\begin{itemize}
			\item Identificador
			\item Nombre de usuario
			\item Correo electrónico
			\item Contraseña
			\item Nombre
			\item Apellidos
			\item Tipo de usuario (oyente, artista)
			\item Dirección
		\end{itemize}
		
	\item Contraseña nueva: \label{rd21}
		\begin{itemize}
			\item Identificador de usuario
			\item Contraseña nueva
		\end{itemize}
		
	\item Datos perfil usuario: \label{rd22}
		\begin{itemize}
			\item Nombre de usuario
			\item Correo electrónico
			\item Contraseña
			\item Nombre
			\item Apellidos
			\item Dirección
		\end{itemize}
		
	\item Identificador del usuario a eliminar: \label{rd23}
		\begin{itemize}
			\item Identificador del usuario
		\end{itemize}
		
		
	\item Datos de un usuario:  \label{rd24}
		\begin{itemize}
			\item Nombre de usuario
		\end{itemize}

	\item Datos de un amigo:  \label{rd25}
		\begin{itemize}
			\item Nombre de usuario
			\item Canción que está escuchando
		\end{itemize}
	\item Datos de artista  \label{rd26}
		\begin{itemize}
			\item Nombre de usuario
			\item Nombre artístico
			\item Número de canciones subidas
			\item Álbumes subidos
		\end{itemize}
	\item Recomendación saliente  \label{rd27}
		\begin{itemize}
			\item Nombre de usuario de salida
			\item Cuerpo de mensaje
			\item Nombre de usuario de entrada
		\end{itemize}
	\item Recomendación entrante  \label{rd28}
		\begin{itemize}
			\item Nombre de usuario de entrada
			\item Cuerpo de mensaje
			\item Nombre de usuario de salida
		\end{itemize}
	\item Recomendación almacenada.  \label{rd29}
		\begin{itemize}
			\item Nombre de usuario que recomienda
			\item Cuerpo de mensaje
			\item Nombre de usuario que es recomendado
		\end{itemize}
	\item Datos de canción que está siendo escuchada  \label{rd30}
		\begin{itemize}
			\item Identificador de canción
			\item Usuario que la está escuchando
		\end{itemize}
	\item Datos del usuario actual  \label{rd31}
		\begin{itemize}
			\item Identificador de usuario
		\end{itemize}
	\item Datos de canciones mejor valoradas  \label{rd32}
		\begin{itemize}
			\item Identificadores de canciones
			\item Identificador de usuario que ha valorado
		\end{itemize}
	\item Datos de usuario que ha realizado valoraciones  \label{rd33}
		\begin{itemize}
			\item Nombre de usuario
		\end{itemize}
		
	\item Valoración dada por un usuario a una canción  \label{rd34}
		\begin{itemize}
			\item identificador de canción
			\item Valoración
		\end{itemize}
		
	\item Datos modificar estado de canción en lista.  \label{rd35}
		\begin{itemize}
			\item Identificador de canción
			\item Identificador de la lista de reproducción
			\item Operación a realizar
		\end{itemize}
		
	\item Identificador de la lista a eliminar  \label{rd36}
		\begin{itemize}
			\item Identificador de la lista de reproducción
		\end{itemize}
		
	\item Datos canciones que están siendo escuchadas  \label{rd37}
		\begin{itemize}
			\item Identificador de la canción
			\item Identificador de usuario que escucha la canción
		\end{itemize}
				
	\item Datos recomendaciones almacenadas  \label{rd38}
		\begin{itemize}
			\item Identificador del usuario que recomienda
			\item Identificador del usuario que recibe la recomendación
			\item Identificador de la canción que es recomendada
		\end{itemize}
		
	\item Datos canciones mejor valoradas \label{rd39}
		\begin{itemize}
			\item Nombre de la canción
			\item Nombre del artista
			\item Valoración
		\end{itemize}

	\item Identificador de un álbum para eliminar \label{rd40}
		\begin{itemize}
			\item Identificador del álbum a eliminar
		\end{itemize}
		
	\item Identificador de álbum introducido \label{rd41}
		\begin{itemize}
			\item Identificador del álbum a eliminar
		\end{itemize}
\end{enumerate}


\subsection{Requisitos funcionales}

\begin{enumerate}[label=\textnormal{RF\arabic*}]

%%%%%%%%%%%%%%%%%% AREA FUNCIONAL ESCUCHAR CANCIONES

    \item Escuchar una canción: un usuario introduce el identificador de la canción, álbum o lista que quiere escuchar.\label{rf1}.
    	\begin{itemize}
			\item Entrada: \ref{rd1}
			\item Consulta: \ref{rd2}, \ref{rd3}, \ref{rd4} 
			\item Salida: \ref{rd14}
		\end{itemize}
		
	 \item Buscar una canción: a través del nombre de una canción, álbum o lista se elige el identificador que se considere el adecuado con los criterios de búsqueda \label{rf2}.
    	\begin{itemize}
			\item Entrada: \ref{rd6}
			\item Consulta: \ref{rd2}, \ref{rd3}, \ref{rd4} 
			\item Salida: \ref{rd7}
		\end{itemize}

	 \item Crear listas de reproducción: un usuario puede crear una lista donde añade canciones y las tiene todas en un mismo lugar. \label{rf3}.
    	\begin{itemize}
			\item Entrada: \ref{rd8}
			\item Almacena: \ref{rd4}
			\item Salida: \ref{rd5}
		\end{itemize}

	 \item Añadir canciones a una lista de reproducción: un usuario puede añadir o eliminar las canciones de su lista a traves del identificador.\label{rf4}.
    	\begin{itemize}
			\item Entrada: \ref{rd35}
			\item Consulta: \ref{rd4}
			\item Actualiza: \ref{rd4}
		\end{itemize}

	 \item Borrar lista: un usuario puede eliminar su lista.\label{rf5}.
    	\begin{itemize}
			\item Entrada: \ref{rd36}
			\item Consulta: \ref{rd4}
			\item Actualiza: \ref{rd4}
		\end{itemize}
		
	 \item Valorar canciones: un usuario puede valorar una canción.\label{rf6}.
    	\begin{itemize}
			\item Entrada: \ref{rd34}
			\item Consulta: \ref{rd9}
			\item Actualiza: \ref{rd9}
		\end{itemize}
		
%%%%%%%%%%%%%%%%%% AREA FUNCIONAL SUBIR CANCIONES
		
	 \item Crear álbum: un artista registra en el sistema un nuevo álbum.\label{rf7}.
    	\begin{itemize}
			\item Entrada: \ref{rd10}
			\item Almacena: \ref{rd2}
			\item Salida: ninguna
		\end{itemize}
		
	 \item Crear canción: un artista introduce en el sistema una nueva canción.\label{rf8}.
    	\begin{itemize}
			\item Entrada: \ref{rd11}
			\item Consulta: \ref{rd2}
			\item Actualiza: \ref{rd3}
			\item Salida: \ref{rd12} 
		\end{itemize}
		
	 \item Ver estadísticas de canción: un artista consulta las estadísticas de una de sus canciones buscándola con su identificador.\label{rf9}.
    	\begin{itemize}
			\item Entrada: \ref{rd13}
			\item Consulta: \ref{rd12}
			\item Salida: \ref{rd15}
		\end{itemize}
		
	 \item Destacar canciones individuales: un artista selecciona una canción a través de su identificador y la añade a su lista de canciones destacadas.\label{rf10}.
    	\begin{itemize}
			\item Entrada: \ref{rd16}
			\item Consulta: \ref{rd3}
			\item Salida: \ref{rd17}
		\end{itemize}
		
	 \item Borrar canciones: un artista introduce en el sistema si quiere eliminar una canción o un álbum y el identificador de su elección y el sistema elimina los datos asociados al mismo.\label{rf11}.
    	\begin{itemize}
			\item Entrada: \ref{rd18}
			\item Consulta: \ref{rd2} \ref{rd3} 
			\item Actualiza: \ref{rd2} \ref{rd3} 
		\end{itemize}
		
%%%%%%%%%%%%%%%%%%%%%%% ÁREA DE ADMINISTRACIÓN DE USUARIOS
		
	 \item Crear usuarios. El sistema guarda la información de un nuevo usuario \label{rf12}.
    	\begin{itemize}
			\item Entrada: \ref{rd19}
			\item Almacena: \ref{rd20}
		\end{itemize}
		
	 \item Recuperar contraseña: \label{rf13}.
    	\begin{itemize}
			\item Entrada: \ref{rd21}
			\item Actualiza: \ref{rd20}
		\end{itemize}
		
	 \item Modificar perfil. El sistema guarda la información modificada por el usuario. \label{rf14}.
    	\begin{itemize}
			\item Entrada: \ref{rd22}
			\item Almacena: \ref{rd20}
		\end{itemize}
		
	 \item Borrar usuarios. Eliminar a un usuario del sistema.\label{rf15}.
    	\begin{itemize}
			\item Entrada: \ref{rd23}
			\item Actualiza: \ref{rd20}
		\end{itemize}
		
%%%%%%%%%%%%%%%%%%%%%%%%% ÁREA SOCIAL
		
	 \item Seguir usuario. Se crea una relación entre dos usuarios, el seguidor estará al tanto de las acciones del usuario al que sigue. \label{rf16}
	 	\begin{itemize}
			\item Entrada: \ref{rd24}
			\item Almacenado: \ref{rd26}
		\end{itemize}
	
	 \item Recomendar canciones a amigos. Un usuario envía a través de un mensaje una canción a otro usuario. \label{rf17}
	 	\begin{itemize}
			\item Entrada: \ref{rd27}
			\item Almacenado: \ref{rd29}
		\end{itemize}
	
	 \item Ver lo que están escuchando los amigos. Se muestra al usuario una lista con las canciones que están escuchando sus amigos en ese momento. \label{rf18}
	 	\begin{itemize}
			\item Entrada: \ref{rd25}
			\item Consulta: \ref{rd37}
			\item Salida: \ref{rd30}
		\end{itemize}

	 \item Ver recomendaciones entrantes. Se muestra al usuario una lista con las recomendaciones que le han hecho sus amigos. \label{rf19}
	 	\begin{itemize}
			\item Entrada: \ref{rd31}
			\item Consulta: \ref{rd38}
			\item Salida: \ref{rd28}
		\end{itemize}

	 \item Canciones mejor valoradas por un usuario. Un usuario solicita una lista de las canciones mejor valoradas proporcionando el identificador de otro usuario. \label{rf20}
	 	\begin{itemize}
			\item Entrada:  \ref{rd33}
			\item Consulta: \ref{rd32}
			\item Salida: \ref{rd39}
		\end{itemize}
		
	 \item Eliminar un álbum existente. Un usuario artista proporciona el identificador de uno de sus álbumes para eliminarlo \label{rf21}
	 	\begin{itemize}
			\item Entrada:  \ref{rd40}
			\item Consulta: \ref{rd41}
		\end{itemize}	
   
\end{enumerate}


\subsection{Validación cruzada de requisitos}
\begin{table}[H]
\begin{center}
\begin{tabular}{|c|c|c|c|}
\hline
	RF & Entrada & Manejo & Salida\\
\hline
	\ref{rf1} & \ref{rd1} & \ref{rd2},\ref{rd3},\ref{rd4} & \ref{rd14}\\
\hline
	\ref{rf2} & \ref{rd6} & \ref{rd4} & \ref{rd7}\\
\hline
	\ref{rf3} & \ref{rd8} & \ref{rd4} & \ref{rd5}\\
\hline
	\ref{rf4} & \ref{rd5} & \ref{rd4} & \\
\hline
	\ref{rf5} & \ref{rd36} & \ref{rd4} & \\
\hline
	\ref{rf6} & \ref{rd5} & \ref{rd9} & \ref{rd9}\\
\hline
	\ref{rf7} & \ref{rd10} & \ref{rd2} & \\
\hline
	\ref{rf8} & \ref{rd11} & \ref{rd2},\ref{rd3} & \ref{rd12}\\
\hline
	\ref{rf9} & \ref{rd13} & \ref{rd12} & \ref{rd15}\\
\hline
	\ref{rf10} & \ref{rd16} & \ref{rd3},\ref{rd10} & \ref{rd17}\\
\hline
	\ref{rf11} & \ref{rd18} & \ref{rd2},\ref{rd3},\ref{rd10} & \\
\hline
	\ref{rf12} & \ref{rd19} & \ref{rd20} & \\
\hline
	\ref{rf13} & \ref{rd21} & \ref{rd20} & \\
\hline
	\ref{rf14} & \ref{rd22} & \ref{rd20} & \\
\hline
	\ref{rf15} & \ref{rd23} & \ref{rd20} & \\
\hline
	\ref{rf16} & \ref{rd24} & \ref{rd25},\ref{rd26} & \\
\hline
	\ref{rf17} & \ref{rd27} & \ref{rd29} & \\
\hline
	\ref{rf18} & \ref{rd25} & \ref{rd37} & \ref{rd30}\\
\hline
	\ref{rf19} & \ref{rd31} & \ref{rd38} & \ref{rd28}\\
\hline
	\ref{rf20} & \ref{rd33} & \ref{rd32} & \ref{rd39}\\
\hline
	\ref{rf21} & \ref{rd40} & \ref{rd41} & \\
\hline
\end{tabular}
\end{center}
\caption{Requisitos funcionales}
\end{table}

\begin{table}[H]
\begin{center}
\begin{tabular}{|c|c|c|c|}
\hline
	RD & Entrada & Manejo & Salida\\
\hline
	\ref{rd1} & \ref{rf1} &  & \\
\hline
	\ref{rd2} &  & \ref{rf1},\ref{rf2},\ref{rf7},\ref{rf8},\ref{rf11} & \\
\hline
	\ref{rd3} &  & \ref{rf1},\ref{rf2},\ref{rf8},\ref{rf10},\ref{rf11} & \\
\hline
	\ref{rd4} &  & \ref{rf1},\ref{rf2},\ref{rf3},\ref{rf4},\ref{rf5} & \\
\hline
	\ref{rd5} &  &  & \ref{rf3} \\
\hline
	\ref{rd6} & \ref{rf2} &  & \\
\hline
	\ref{rd7} &  &  & \ref{rf2}\\
\hline
	\ref{rd8} & \ref{rf3} &  & \\
\hline
	\ref{rd9} &  & \ref{rf6} & \\
\hline
	\ref{rd10} & \ref{rf7} &  & \\
\hline
	\ref{rd11} & \ref{rf8} &  & \\
\hline
	\ref{rd12} &  & \ref{rf9} & \ref{rf8}\\
\hline
	\ref{rd13} & \ref{rf9} &  & \\
\hline
	\ref{rd14} &  &  & \ref{rf1}\\
\hline
	\ref{rd15} &  &  & \ref{rf9}\\
\hline
	\ref{rd16} & \ref{rf10} &  & \\
\hline
	\ref{rd17} &  &  & \ref{rf10} \\
\hline
	\ref{rd18} & \ref{rf11} &  & \\
\hline
	\ref{rd19} & \ref{rf12} &  & \\
\hline
	\ref{rd20} &  & \ref{rf12},\ref{rf13},\ref{rf14},\ref{rf15} & \\
\hline
	\ref{rd21} & \ref{rf13} &  & \\
\hline
	\ref{rd22} & \ref{rf14} &  & \\
\hline
	\ref{rd23} & \ref{rf15} &  & \\
\hline
	\ref{rd24} & \ref{rf16} &  & \\
\hline
	\ref{rd25} & \ref{rf18} & & \\
\hline
	\ref{rd26} &  & \ref{rf16} & \\
\hline
	\ref{rd27} & \ref{rf17} &  & \\
\hline
	\ref{rd28} &  &  & \ref{rf19}\\
\hline
	\ref{rd29} &  & \ref{rf17} & \\
\hline
	\ref{rd30} &  &  & \ref{rf18} \\
\hline
	\ref{rd31} & \ref{rf19} &  & \\
\hline
	\ref{rd32} &  & \ref{rf20} & \\
\hline
	\ref{rd33} & \ref{rf20} &  & \\
\hline
	\ref{rd34} & \ref{rf6} &  & \\
\hline
	\ref{rd35} & \ref{rf4} &  & \\
\hline
	\ref{rd36} & \ref{rf5} &  & \\
\hline
	\ref{rd37} &  & \ref{rf18} & \\
\hline
	\ref{rd38} &  & \ref{rf19} & \\
\hline
	\ref{rd39} &  & & \ref{rf20} \\
\hline
	\ref{rd40} & \ref{rf21} & &  \\
\hline
	\ref{rd41} &  & \ref{rf21} &  \\
\hline
\end{tabular}
\end{center}
\caption{Requisitos de datos}
\end{table}

\section{Diseño.}
\subsection{Esquema de caja negra.}
  \begin{center}
  \shadowbox{\includegraphics[width=0.6\textwidth]{imagenes/caja_negra.png}\par\vspace{1cm}}
  \end{center}
\subsection{Esquema armazón DFD0.}
  \begin{center}
  \shadowbox{\includegraphics[width=0.6\textwidth]{imagenes/DFD0.png}\par\vspace{1cm}}
  \end{center}
\subsection{Esquemas externos del DFD0.}
	%No se que esquemas son estos
	\subsubsection{EE0 Módulo social}
  \begin{center}
  \shadowbox{\includegraphics[width=0.6\textwidth]{imagenes/EE0_social.png}\par\vspace{1cm}}
  \end{center}
	
\subsection{Refinamientos.}



%%%%%%%%%%%%%%%%% MODULO ESCUCHAR CANCIONES
\subsubsection{Módulo escuchar canciones.}
\paragraph{DFD1}
\begin{center}
\shadowbox{\includegraphics[width=0.8\textwidth]{imagenes/DFD1_escuchar_musica.png}\par\vspace{1cm}}
\end{center}
\paragraph{DFD2}
\subparagraph{Primer plano de refinamiento.}
\begin{center}
\shadowbox{\includegraphics[width=0.8\textwidth]{imagenes/DFD2_1-1_escuchar_musica.png}\par\vspace{1cm}}    
\end{center}
\subparagraph{Segundo plano de refinamiento.}
\begin{center}
\shadowbox{\includegraphics[width=0.8\textwidth]{imagenes/DFD2_1-2_escuchar_musica.png}\par\vspace{1cm}}    
\end{center}
\paragraph{Esquema externo.}
\begin{center}
\shadowbox{\includegraphics[width=0.8\textwidth]{imagenes/EE_escuchar_musica.png}\par\vspace{1cm}}
\end{center}

%%%%%%%%%%%%%%%%% MODULO GESTIONAR CANCIONES
\subsubsection{Módulo gestionar canciones.}
\paragraph{DFD1}
\begin{center}
\shadowbox{\includegraphics[width=0.8\textwidth]{imagenes/DFD1_Gestion_de_canciones.pdf}\par\vspace{1cm}}
\end{center}
\paragraph{EE1}
\begin{center}
\shadowbox{\includegraphics[width=0.8\textwidth]{imagenes/EE1_Gestion_de_canciones.pdf}\par\vspace{1cm}}
\end{center}
\paragraph{DFD2}
\begin{center}
\shadowbox{\includegraphics[width=0.8\textwidth]{imagenes/DFD2_Gestion_de_canciones.pdf}\par\vspace{1cm}}    
\end{center}
\paragraph{Esquema externo.}
\begin{center}
\shadowbox{\includegraphics[width=0.8\textwidth]{imagenes/EE2_Gestion_de_canciones.pdf}\par\vspace{1cm}}
\end{center}

%%%%%%%%%%%%%%%%% MODULO GESTIÓN DE USUARIOS
\subsubsection{Módulo gestión de usuarios.}
\paragraph{DFD1}
\begin{center}
\shadowbox{\includegraphics[width=0.8\textwidth]{imagenes/DFD1_Gestion_de_usuarios.pdf}\par\vspace{1cm}}
\end{center}
\paragraph{DFD2}
\begin{center}
\shadowbox{\includegraphics[width=0.8\textwidth]{imagenes/DFD2_Gestion_de_usuarios.pdf}\par\vspace{1cm}}    
\end{center}
\paragraph{Esquema externo.}
\begin{center}
\shadowbox{\includegraphics[width=1\textwidth]{imagenes/EE2_usuarios.pdf}\par\vspace{1cm}}
\end{center}


%%%%%%%%%%%%%%%%% MODULO SOCIAL
\subsubsection{Módulo social.}
\paragraph{DFD1}
\begin{center}
\shadowbox{\includegraphics[width=0.6\textwidth]{imagenes/DFD1_social.png}\par\vspace{1cm}}
\end{center}	
\paragraph{EE1}
\begin{center}
\shadowbox{\includegraphics[width=0.6\textwidth]{imagenes/EE1_social.png}\par\vspace{1cm}}
\end{center}
\paragraph{DFD2}
\subparagraph{Primer plano de refinamiento.}
\begin{center}
\shadowbox{\includegraphics[width=0.6\textwidth]{imagenes/DFD2_4-1_social.png}\par\vspace{1cm}}    
\end{center}
\newpage
\subparagraph{Segundo plano de refinamiento.}
\begin{center}
\shadowbox{\includegraphics[width=0.6\textwidth]{imagenes/DFD2_4-2_social.png}\par\vspace{1cm}}    
\end{center}
\paragraph{Esquema externo.}
\begin{center}
\shadowbox{\includegraphics[width=1.0\textwidth]{imagenes/EE_social.png}\par\vspace{1cm}}
\end{center} 


%%%%%%%%%%%%%%%%% UNIONES DE ESQUEMAS
\subsubsection{Unión de módulos DFD1.}
\begin{center}
\shadowbox{\includegraphics[width=1.0\textwidth]{imagenes/DFD1.pdf}\par\vspace{1cm}}
\end{center}
\subsubsection{Unión de módulos DFD2.}
\begin{center}
\shadowbox{\includegraphics[width=1.0\textwidth]{imagenes/DFD2.pdf}\par\vspace{1cm}}
\end{center}

\subsection{Esquema entidad relación.}
\begin{center}
\shadowbox{\includegraphics[width=1.0\textwidth]{imagenes/ER.pdf}\par\vspace{1cm}}
\end{center}




\subsection{Paso a tablas.}

%Tabla usuario.

\begin{table}[H]
\centering
\begin{tabular}{|c|c|}
\hline
\multicolumn{2}{|c|}{\textbf{Tabla Usuarios}} \\ \hline
Clave primaria & ID usuario \\ \hline
Resto de atributos & \begin{tabular}[c]{@{}c@{}}nombre usuario, nombre artístico,\\ email, contraseña, nombre, apellidos, tipo, pais.\end{tabular} \\ \hline
\end{tabular}
\end{table}

%Tabla canción.

\begin{table}[H]
\centering
\begin{tabular}{|c|c|}
\hline
\multicolumn{2}{|c|}{\textbf{Tabla Canción}} \\ \hline
Clave primaria & ID canción \\ \hline
Resto de atributos & \begin{tabular}[c]{@{}c@{}}nombre, genero, duración, fecha, ruta audio,\\ numero de reproducciones, valoración.\end{tabular} \\ \hline
\end{tabular}
\end{table}

%Tabla album.

\begin{table}[H]
\centering
\begin{tabular}{|c|c|}
\hline
\multicolumn{2}{|c|}{\textbf{Tabla Álbum}} \\ \hline
Clave primaria & ID album \\ \hline
Resto de atributos & \begin{tabular}[c]{@{}c@{}}nombre, fecha, numero de canciones, duración\\ 
	\end{tabular} \\ \hline
\end{tabular}
\end{table}

%Tabla lista.
	
	
\begin{table}[H]
\centering
\begin{tabular}{|c|c|}
\hline
\multicolumn{2}{|c|}{\textbf{Tabla Lista}} \\ \hline
Clave primaria & ID lista \\ \hline
Resto de atributos & \begin{tabular}[c]{@{}c@{}}nombre, fecha, duración.\end{tabular} \\ \hline
\end{tabular}
\end{table}
	
%Tabla amigo.

\begin{table}[H]
\centering
\begin{tabular}{|c|c|}
\hline
\multicolumn{2}{|c|}{\textbf{Tabla Amigo}} \\ \hline
Clave primaria & ID usuario1, ID usuario2. \\ \hline
\multicolumn{2}{|c|}{Ambas son claves externas a ID usuario de la tabla usuario} \\ \hline
\end{tabular}
\end{table}
	
	
	
%Tabla recomienda.
	
\begin{table}[H]
\centering
\begin{tabular}{|c|c|}
\hline
\multicolumn{2}{|c|}{\textbf{Tabla Recomienda}} \\ \hline
Clave primaria & ID usuario1, ID usuario2, ID cancion, fecha. \\ \hline
\multicolumn{2}{|c|}{\begin{tabular}[c]{@{}c@{}}ID usuario1 e ID usuario2 como claves externas a la tabla amigo.\\ ID cancion como clave externa a la tabla cancion.\end{tabular}

	} \\ \hline
\end{tabular}
\end{table}
	
%Tabla escucha.
	
\begin{table}[H]
\centering
\begin{tabular}{|c|c|}
\hline
\multicolumn{2}{|c|}{\textbf{Tabla Escucha}} \\ \hline
Clave primaria & ID cancion, ID usuario, fecha \\ \hline
\multicolumn{2}{|c|}{
	\begin{tabular}[c]{@{}c@{}}
	  ID usuario como clave externa a la tabla usuario.\\
	  ID canción como clave externa a la tabla canción.
	\end{tabular}} \\ \hline
\end{tabular}
\end{table}
	
%Tabla valora.
	
\begin{table}[H]
\centering
\begin{tabular}{|c|c|}
\hline
\multicolumn{2}{|c|}{\textbf{Tabla Valora}} \\ \hline
Clave primaria & ID usuario, ID cancion \\ \hline
\multicolumn{2}{|c|}{
	\begin{tabular}[c]{@{}c@{}}
	  ID usuario como clave externa a la tabla usuario.\\
	  ID canción como clave externa a la tabla canción.
	\end{tabular}} \\ \hline
	Resto de atributos & puntuación \\ \hline
\end{tabular}
\end{table}
	
%Tabla pertenece.
	
\begin{table}[H]
\centering
\begin{tabular}{|c|c|}
\hline
\multicolumn{2}{|c|}{\textbf{Tabla Pertenece}} \\ \hline
Clave primaria & ID lista, ID cancion \\ \hline
\multicolumn{2}{|c|}{
	\begin{tabular}[c]{@{}c@{}}
	  ID lista como clave externa a la tabla lista.\\
	  ID canción como clave externa a la tabla canción.
	\end{tabular}} \\ \hline
\end{tabular}
\end{table}
	
%Tabla crea.
	
\begin{table}[H]
\centering
\begin{tabular}{|c|c|}
\hline
\multicolumn{2}{|c|}{\textbf{Tabla Crea}} \\ \hline
Clave primaria & ID lista, ID usuario \\ \hline
\multicolumn{2}{|c|}{
	\begin{tabular}[c]{@{}c@{}}
	  ID lista como clave externa a la tabla lista.\\
	  ID usuario como clave externa a la tabla usuario.
	\end{tabular}} \\ \hline
\end{tabular}
\end{table}

%Tabla Formado.
	
\begin{table}[H]
\centering
\begin{tabular}{|c|c|}
\hline
\multicolumn{2}{|c|}{\textbf{Tabla Formado}} \\ \hline
Clave primaria & ID álbum, ID cancion \\ \hline
\multicolumn{2}{|c|}{
	\begin{tabular}[c]{@{}c@{}}
	  ID álbum como clave externa a la tabla álbum.\\
	  ID canción como clave externa a la tabla canción.
	\end{tabular}} \\ \hline
\end{tabular}
\end{table}

%Tabla Desarrolla.
	
\begin{table}[H]
\centering
\begin{tabular}{|c|c|}
\hline
\multicolumn{2}{|c|}{\textbf{Tabla Desarrolla}} \\ \hline
Clave primaria & ID usuario, ID cancion \\ \hline
\multicolumn{2}{|c|}{
	\begin{tabular}[c]{@{}c@{}}
	  ID usuario como clave externa a la tabla usuario.\\
	  ID canción como clave externa a la tabla canción.
	\end{tabular}} \\ \hline
\end{tabular}
\end{table}

%Tabla Produce.
	
\begin{table}[H]
\centering
\begin{tabular}{|c|c|}
\hline
\multicolumn{2}{|c|}{\textbf{Tabla Produce}} \\ \hline
Clave primaria & ID álbum, ID usuario \\ \hline
\multicolumn{2}{|c|}{
	\begin{tabular}[c]{@{}c@{}}
	  ID álbum como clave externa a la tabla álbum.\\
	  ID usuario como clave externa a la tabla usuario.
	\end{tabular}} \\ \hline
\end{tabular}
\end{table}

%Tabla Destacadas.
	
\begin{table}[H]
\centering
\begin{tabular}{|c|c|}
\hline
\multicolumn{2}{|c|}{\textbf{Tabla Destacadas}} \\ \hline
Clave primaria & ID usuario, ID canción \\ \hline
\multicolumn{2}{|c|}{
	\begin{tabular}[c]{@{}c@{}}
	  ID usuario como clave externa a la tabla usuario.\\
	  ID canción como clave externa a la tabla canción.
	\end{tabular}} \\ \hline
\end{tabular}
\end{table}
	
\section{Implementación.}
\subsection{Diseño físico de la base de datos.}
\newpage

\subsubsection{Sentencias SQL.}
\lstinputlisting[language=SQL]{SQL/SQL_pdf.sql}

\newpage
\subsection{Implementación modulo escuchar música.}
\subsubsection{Implementación del disparador.}
	El disparador se encarga de controlar la inserción de una tupla en la tabla pertenece. Esta tabla es la encargada de guardar las canciones que pertenecen a una lista determinada. El trabajo del disparador es que cuando una canción pertenezca a mas de dos lista diferentes muestra un mensaje 
	
	El funcionamiento consiste en contar cuantas veces aparece la misma canción en la tabla pertenece (cada vez que aparezca quiere decir que esta en otra lista) antes de la inserción de esta a una nueva lista. Si es mayor que uno muestra el mensaje. Muestra el mensaje cuando es mayor que uno y no dos por que el disparador es un BEFORE INSERT y a ese uno hay que sumarle la inserción próxima.

\begin{lstlisting}[language=SQL]
CREATE OR REPLACE TRIGGER pedro_trigg
BEFORE INSERT ON pertenece
FOR EACH ROW
DECLARE
num_canciones number := 0;
BEGIN
    SELECT count(id_cancion) 
    INTO num_canciones 
    FROM pertenece  
    where id_cancion = :new.id_cancion;
    
	if (num_canciones > 1) THEN
		DBMS_OUTPUT.PUT_LINE('La cancion pertenece a mas de dos listas');
	END IF;    
END;
\end{lstlisting}

\subsubsection{Implementación de añadir canción a una lista.}
	La implementación ha sido implementada en Java y la interfaz es mediante lineas de comandos. Se muestra un menú y el usuario tiene que elegir una opción. 
	
	El acceso a la base de datos se realiza JDBC, por eso la implementación está escrita en Java. Para el desarrollo he usado JDeveloper.


\lstinputlisting[language=Java,frame=single,tabsize=2,showstringspaces=false,basicstyle=\small]{implementaciones/pedro_aniadir_canciones_a_lista.java}

\newpage
\subsection{Implementación modulo social.}
\subsubsection{Implementación del disparador.}
	Este disparador controla la inserción de una tupla en la tabla recomendación. Es decir comprueba una condición cuando un usuario recomienda una canción a otro usuario. Dicha condición consiste en que un usuario no puede recomendar una canción si no la ha escuchado previamente.
	
	Básicamente, el funcionamiento consiste en comprobar la tabla escucha (donde reside el registro de reproducciones que han realizado los usuarios) y mirar si el usuario que recomienda ha reproducido la canción que se recomienda canción. Si no encuentra ningún resultado, levanta una excepción.

\begin{lstlisting}[language=SQL]
create or replace TRIGGER dario_trigg
BEFORE INSERT ON recomienda
FOR EACH ROW
DECLARE
no_escuchada EXCEPTION;
existe number;
BEGIN
SELECT count(*) INTO existe FROM escucha 
WHERE id_cancion = :new.id_cancion and id_usuario = :new.id_usuario1;
    IF (existe = 0) THEN
           RAISE no_escuchada;
    END IF;
    EXCEPTION
    WHEN no_escuchada THEN
    RAISE_APPLICATION_ERROR(-20000,'No has escuchado la cancion');
END;
\end{lstlisting}
\subsubsection{Implementación de un proceso.}
	El proceso acordado para la implementación consiste mostrar que están escuchando mis amigos. Debido a la dificultad de mostrar lo que se está reproduciendo en un instante, se decidió mostrar las reproducciones realizadas ``hoy'', es decir, en un día completo.
	
	El proceso ha sido implementado en Python y la interfaz es línea de comandos (CLI), debido a la falta de tiempo y el poco conocimiento acerca de interfaces gráficas. 
	
	La conexión a la base de datos se realiza mediante el paquete ``cx\_Oracle'', que nos permite realizar las consultas y manejar los datos consultados.


\lstinputlisting[language=Python,frame=single,tabsize=2,showstringspaces=false]{implementaciones/dario_ver_canciones_amigos.py}

\newpage
\subsection{Valoración de canciones.}
\subsubsection{Implementación del disparador.}
El disparador que tenemos que hacer tiene que comprobar que cuando vamos a valorar una canción, ésta ha sido escuchada al menos una vez por el usuario que valora. Para hacer esto implementamos un disparador que se ejecuta antes de cada operación \texttt{INSERT} en la tabla \texttt{valora}.

Este disparador lo que hace es comprobar el número de escuchas que ha tenido la canción a valorar por el usuario que valora. Si este número es 0 (no ha escuchado nunca la canción) lanza una excepción con un mensaje que avisa al usuario que no puede ejecutarse esta operación.

El código del disparador implementado es el siguiente:

\lstinputlisting[language=SQL, firstline=29, lastline=45, tabsize=2,showstringspaces=false]{SQL/javier_trigger.sql}

Para probar que este disparador funciona correctamente hacemos las siguiente operaciones:

- Vamos a intentar crear una valoración de una canción que no hemos escuchado
\begin{verbatim}
INSERT INTO valora(id_usuario, id_cancion, puntuacion) 
values('UN000003', 'C0000001', 5.0);
\end{verbatim}
- Nos da un mensaje de error, ya que no hemos escuchado la canción (el disparador genera una excepción)
\begin{verbatim}
INSERT INTO valora(id_usuario, id_cancion, puntuacion) 
	values('UN000003', 'C0000001', 5)
Informe de error -
ORA-20001: No puedes valorar una cancion que no has escuchado
ORA-06512: en "X7151952.JAVIER_TRIGG", línea 11
ORA-04088: error durante la ejecución del disparador 'X7151952.JAVIER_TRIGG'
\end{verbatim}

- Insertamos ahora una escucha por el usuario que desea valorar

\begin{verbatim}
INSERT INTO escucha(id_usuario, id_cancion, fecha) 
	values('UN000003', 'C0000001', to_date('19/01/2019','DD/MM/YYYY'));
\end{verbatim}
- Si insertamos ahora una valoración funciona sin problemas.
\begin{verbatim}
1 fila insertadas
\end{verbatim}

- Devolvemos la BD a su estado original
\begin{verbatim}
DELETE FROM escucha where id_usuario = 'UN000003' and ID_CANCION = 'C0000001';
DELETE FROM valora where id_usuario = 'UN000003' and ID_CANCION = 'C0000001';
commit;
\end{verbatim}

\subsubsection{Implementación de un proceso.}
	
	El proceso a implementar según lo acordado con el profesor es el de valorar canciones. Para ello vamos a desarrollar una aplicación que muestre las canciones del sistema y permita valorarlas. Si la canción no ha sido escuchada por el usuario da un error advirtiendo al usuario que no puede valorar una canción que no ha escuchado. Además muestra un error si la canción ya ha sido valorada por el usuario.
	
	El proceso ha sido implementado en Python y Javascript. Se ha diseñado una interfaz gráfica basada en tecnologías web que hace consultas a una API creada en Python con el microframework Flask. La conexión a la base de datos se realiza mediante el paquete ``cx\_Oracle'', que nos permite realizar las consultas y manejar los datos consultados.
	
	El código del servidor es el siguiente:
	
\lstinputlisting[language=Python,tabsize=1,showstringspaces=false]{implementaciones/implementacion_javier/servicio_api.py}

	El código Javascript que consume la API es el siguiente:
	
\lstinputlisting[language=JavaScript,tabsize=1,showstringspaces=false]{implementaciones/implementacion_javier/main.js}

Para actualizar las valoraciones de cada canción después de que se haya añadido una valoración nueva se ha implementado un nuevo disparador que actualiza la valoración en la tabla canción después de cada INSERT en la tabla valora.

\lstinputlisting[language=SQL,tabsize=1,showstringspaces=false]{SQL/valora_trigger.sql}

	La aplicación tiene la siguiente apariencia:
\begin{figure}[H]
\centering
\includegraphics[scale=0.40]{implementaciones/implementacion_javier/imagenes/intro.png}
\end{figure}

\begin{figure}[H]
\centering
\includegraphics[scale=0.40]{implementaciones/implementacion_javier/imagenes/lista_canciones.png}
\end{figure}

\begin{figure}[H]
\centering
\includegraphics[scale=0.40]{implementaciones/implementacion_javier/imagenes/valoracion.png}
\end{figure}

\begin{figure}[H]
\centering
\includegraphics[scale=0.40]{implementaciones/implementacion_javier/imagenes/duplicada.png}
\end{figure}

\begin{figure}[H]
\centering
\includegraphics[scale=0.40]{implementaciones/implementacion_javier/imagenes/no_escuchada.png}
\end{figure}


\end{document}
